\documentclass[11pt,article,oneside]{memoir} %{{{
% based on Kieran Healy's syllabus templates
% https://github.com/kjhealy/latex-custom-kjh
\usepackage{booktabs}

\usepackage{org-preamble-pdflatex}
\usepackage[margin=1.2in]{geometry}

\usepackage{enumitem}
\setlist{nolistsep}

\setlength{\parskip}{10pt}
\setlength{\parindent}{0pt}

%}}}
% Definitions %{{{
\def\myauthor{Author}
\def\mytitle{Title}
\def\mycopyright{\myauthor}
\def\mykeywords{}
\def\mybibliostyle{plain}
\def\mybibliocommand{}
\def\mysubtitle{}
\def\myaffiliation{Binghamton University}
\def\myaddress{\url{https://binghamton.zoom.us/my/yangkc}}
\def\myemail{yangkc@binghamton.edu}
\def\myweb{https://www.kaichengyang.me/kaicheng}
\def\myphone{}
\def\myversion{}
\def\myrevision{}

\def\myauthor{Kai-Cheng Yang}
\def\mykeywords{Social Media, Data Science}
\def\mysubtitle{Syllabus}
\def\mytitle{{\normalsize CS 415/515} \\ \HUGE{} Social Media Data Science Pipelines}

%%\chapterstyle{article-3}
%\pagestyle{kjh}

\def\ind{\hangindent=1 true cm\hangafter=1 \noindent}
\def\labelitemi{$\cdot$}

\chapterstyle{article-4}  % alternative styles are defined in latex-custom-kjh/needs-memoir/

%}}}
\begin{document} %{{{

\title{\LARGE \mytitle} %{{{
\date{Last updated: \today}

\published{\sffamily Fall 2025}

\maketitle

Instructor: \myauthor \\
Email: \myemail\\
Office: G06A, Engineering Building \\
Lecture: Tuesday \& Thursday 9:45am - 11:15am \\
Office hours: Tuesday 12:30pm - 2:30pm and by appointment \\
% {Homepage: \url{https://github.com/yangkc/social-media-ds-course}}

Teaching Assistant: None \\

%}}}
\section{Course Description}%{{{

The focus of this course is on applying data science techniques to large-scale social media.
The topics covered include large-scale data collection and management, exploratory analysis and measurement techniques, data visualization, hypothesis testing and statistical modeling, and predictive, real-time analytics.
Students will build an end-to-end analysis pipeline and use it to answer questions about online events as they occur.
The goal of the class is to provide students with a methodological toolbox, the technical skills to make use of these tools, and the experience of using them on real world data.

%}}}

\section{Useful Links}

The course website is \url{https://yangkclab.github.io/social-media-ds-course}, which contains the schedule and other course resources.
Please check the website regularly for updates.

The up-to-date syllabus can be downloaded from \url{https://github.com/yangkclab/social-media-ds-course/blob/main/syllabus/syllabus.pdf}.

\section{Credit Hours}

This course is cross-listed as CS 415 and CS 515.

Both sections are 3 credit hours, which means that in addition to the scheduled lectures/discussions, students are expected to do at least 6.5 hours of course-related work each week during the semester.
This includes things like: completing assigned readings, participating in lab sessions, studying for tests and examinations, preparing written assignments, and other tasks that must be completed to earn credit in the course.

\section{Course Objectives}%{{{

This course is designed to provide a solid foundation and background in performing data science on social media.
In particular, upon successful completion of this course, you will be able to:

\begin{itemize}
    \item Build a continuous data system for social media.
    \item Manage collected data.
    \item Design and execute various measurements on social media.
    \item Model and analyze online behavior via social media.
    \item Create visualizations that help understand social media phenomena.
\end{itemize}

%}}}
% \section{Communication} %{{{

% %}}}

\section{Prerequisites and Co-requisites}%{{{
\label{sec:Prerequisites}

\begin{itemize}
    \item CS 350 Operating Systems
    \item CS 375 Design \& Analysis Algorithm
    \item MATH 327 Probability with Stat Methods or equivalent
    \item Know at least one programming language well
\end{itemize}

%}}}

\section{Relationship with ABET}%{{{

\begin{itemize}
    \item Student Outcome 5 (Function effectively as a member or leader of a team engaged in
    activities appropriate to the program's discipline): All programming projects are required
    team projects of 2-3 students.
    \item Exposure to information management: This course is a designated course for this
    requirement.
\end{itemize}

%}}}

\section{Textbook and Reference Books}

Material in this class is delivered via lecture and reading research papers; there is no
textbook.

\section{Course Format and Topics}

This class combines lectures with research paper reading and discussions.

The lectures will cover the fundamentals of data science on social media.
The following is a non-exhaustive list of topics that will be covered in the lectures:

\begin{itemize}
    \item What is Data Science and what does social media have to do with it?
    \item Data collection
    \item Social media data formats
    \item Social media data management with RDBMS/NoSQL
    \item Applications of probability and statistics, with an emphasis on hypothesis testing
    \item Applications of Machine Learning
    \item Visualization
\end{itemize}

The reading materials will be recent research papers that are related to the topics covered in the lectures.
The main topics that will be covered in the reading materials will be:

\begin{itemize}
    \item Dataset and data collection
    \item Algorithmic bias
    \item Inauthentic behaviors
    \item Ethics and data access
    \item Generative AI and social media
\end{itemize}


\section{Lecture Notes and Supplemental Materials}
\begin{itemize}
    \item Lecture notes will be provided via PDFs or other formats delivered in class.
    \item All paper reading assignments will be made available via Brightspace.
\end{itemize}


\section{Assignments}

\begin{itemize}
    \item Paper readings. The best way to start understanding what you can do with data science is
    to explore the state-of-the art. The best way to do that is to read research papers and that
    is what we will do in this class. There will be regular paper readings. For each paper,
    there will potentially be an in-class quiz.
    It is expected that all students come prepared (i.e., read the paper) and participate in the
    discussion to the best of their abilities.
    \item There will be three projects. Each project has three parts: a proposal, an implementation, and a report.
    Projects are to be completed in groups of 2-3 students.
\end{itemize}

Important notes about assignments:

\begin{itemize}
    \item Late assignments may sometimes be accepted with penalty, which will typically be 5\%
    per day late (including weekends and holidays). We will not accept assignments more
    than 5 days after the due date unless there is a very compelling reason.
    \item For project submissions, please ensure your programs are clear, well-structured, and easy to evaluate.
    All programming assignments should have:
    \begin{itemize}
        \item An adequate explanation of the design of your program.
        \begin{itemize}
            \item You should be prepared to answer good-faith, technical questions asked about
            your design and implementation during 1:1 sessions with the instructor.
        \end{itemize}
        \item Documented code:
        \begin{itemize}
            \item Ideally, you use whatever documentation tools are available in the language you
            decide to implement in, but at minimum, all modules, classes, and functions
            should have a documentation header that explains what the code does.
        \end{itemize}
    \end{itemize}
    \item Grading disputes, regrading and missing grades.
    \begin{itemize}
        \item Should you dispute any grading, please be aware that we will not re-grade any
single issue you have. Instead, your work will be re-evaluated from scratch. The
new grade may be higher, lower, or stay the same. This new grade will not be
changed.
        \item No regrading can be requested two weeks after the date when graded work is
        returned to students.
        \item The scores of your assignments will be made available to you after the
assignments are graded.
    \end{itemize}
\end{itemize}

\section{Method of Assessment}%{{{

The following percentage weights will be used to assess student work:

\begin{itemize}
    \item Paper reading quizzes are worth 15\%.
    \item Three programming assignments (projects) 85\% split evenly across all three projects.
    % \item Any group containing graduate student groups will have additional requirements for each project.
\end{itemize}


%}}}

\section{Grading Determination}\label{sec:grading_tentative_}%{{{

Your final grade for this course is largely based on your performance relative to the performance
of other students in the class. In other words, if your work is consistently better than average, you
are likely to receive an A.
The specific break down of grades is:

\begin{itemize}
    \item A: 100--90
    \item B: 89--80
    \item C: 79--70
    \item D: 69--60
    \item F: 59--0
\end{itemize}

There are no +/- grades.

\section{Academic Honesty Expectations and Violation Penalty}

\begin{itemize}
    \item Cheating on quizzes of any kind, including, but not limited to, the use of electronic devices, ``cheat sheets,'' or looking at another student's quiz are considered instances of cheating and will be reported as Category 1 academic dishonesty violation. You will also receive a one letter grade deduction (e.g., from A to B). More than one incident of cheating of any kind will result in an F for the entire course.
    \item The School of Computing at Binghamton wrote a letter to all computer science students about the importance of academic honesty. The letter is available at \url{https://www.binghamton.edu/watson/about/academic-honesty.html}.
    \item Please review the academic honesty document and make sure that you understand it!
    \item Each assignment must include the following statement, verbatim, followed by your group members' names in a file called "HONESTY.md":
    \begin{quote}
        ``We have done this assignment completely on our own. We have not copied it, nor have we given our solution to anyone else. We understand that if we are involved in plagiarism or cheating, we will have to sign an official form that we have cheated and that this form will be stored in our official university records. We also understand that we will receive a grade of 0 for the involved assignment and our grade will be reduced by at least one level (e.g., from A to B) for our offense, and that we will receive a grade of ``F'' for the course for any additional offense of any kind.''
    \end{quote}
    Failure to submit your HONESTY.md file with the above text, verbatim, will result in your project not being graded and you receiving a 0 for the submission.
    \item For this course, programming assignments (projects) are all team projects. Certain open-source tools/software are permitted to be used (see the description of each project for details). Used open-source tools/software must be clearly acknowledged in the submitted project report.
    \item Additionally, each project submission must include a statement of contribution, which describes which group members did what part of the assignment in a file called "CREDITS.md". Failure to submit your CREDITS.md file will result in your submission not being graded and receiving a 0 for the submission.
    \item The use of generative AI tools is allowed for this course given that students follow the guidelines detailed below.
    Violation of the guidelines and inappropriate use of generative AI tools will be considered cheating and will be reported as Category 1 academic dishonesty violation.
    More than one incident of cheating of any kind will result in an F for the entire course.
    % \item Your assignments will not be graded unless the statement above is present and is followed by your signature.
\end{itemize}


\section{Generative AI policy}

Since this is not a generative AI course, we take a moderate stance on the use of generative AI tools: they are allowed but not encouraged.
Students are encouraged to complete coursework independently to maximize learning outcomes.
However, we recognize the potential benefits of generative AI tools in the learning process.
Below are guidelines for students who choose to use these tools.

Generative AI tools include but are not limited to:
\begin{itemize}
    \item LLMs, such as ChatGPT, Claude, and Gemini
    \item Coding assistants, such as Cursor, GitHub Copilot, and Claude Code.
\end{itemize}

Students are allowed to use generative AI tools to:

\begin{itemize}
    \item Enhance their understanding of course material, such as gathering information and explaining concepts.
    \item Clarify ideas and polish their writing.
    \item Assist with project implementation.
\end{itemize}

Students should be aware that generative AI outputs can be erroneous, and they are fully responsible for verifying accuracy.
Additionally, AI outputs may not meet assignment requirements.
Since students are ultimately accountable for their submitted work, the quality will be reflected in their grade.

Students who choose to use generative AI tools must include an AI usage statement in their submitted assignments.
The statement should include the following information:

\begin{itemize}
    \item The generative AI tool used.
    \item The prompt used to generate the output.
    \item How the AI-generated content was integrated into the assignment.
\end{itemize}

The following are not allowed:

\begin{itemize}
    \item Using generative AI tools to generate entire assignments without modification. Submitted work should reflect the student's own understanding, ideas, and effort.
    \item Any use of generative AI tools without properly acknowledging it in the AI usage statement.
\end{itemize}


\section{Managing Stress}

If you are experiencing undue personal or academic stress at any time during the semester or need to talk with someone about a personal problem or situation, I encourage you to seek support as soon as possible.
I am available to talk with you about stresses related to your work in my class.
Additionally, I can assist you in reaching out to any one of a wide range of campus resources, including:

\begin{itemize}
    \item Dean of Students Office: 607-777-2804
    \item Decker Student Health Services Center: 607-777-2221
    \item University Police: On campus emergency, 911 or 607-777-2222
    \item University Counseling Center: 607-777-2772
    \item Interpersonal Violence Advocate: 607-777-2804
    \item Harpur Advising: 607-777-6305
    \item Office of International Student \& Scholar Services: 607-777-2510
    \item Ombudsman: 607-777-2388
    \item Services for Students with Disabilities: 607-777-2686 (Voice, TTY)
\end{itemize}

\section{Title IX Policy}

In the event that you choose to write or speak about experiencing or surviving sexual violence (including sexual harassment, dating, and domestic violence), sexual assault, stalking, and rape, please keep in mind that federal and state laws require that, as your instructor, notify the Title IX Coordinator.
He will contact you and provide you with on and off campus resources and discuss your options with you.
If you would like to disclose your experience confidentially, you can contact the University Counseling Center, Decker Student Health Services, Harpur's Ferry, Ombudsman, or the Binghamton University Interfaith Council (BUIC).

For more information, please go to the VARCC website (\url{https://www.binghamton.edu/centers/varcc/index.html}), or the Title IX at Binghamton University website (\url{https://www.binghamton.edu/services/title-ix/index.html}).

\section{Mental Health}

Diminished mental health, including significant stress, mood changes, excessive worry, or problems with eating and/or sleeping can interfere with optimal academic performance.
The source of symptoms might be largely related to your coursework; if so, I invite you to speak with me (or your other faculty) directly.
However, problems with relationships, family worries, loss, or a personal struggle or crisis can also contribute to decreased academic performance and may require additional professional support.
Binghamton University provides a variety of support resources: the Dean of Students Office and University Counseling Center offer coaching on ways to reduce the impact on your grades.
Both of these resources can help you manage personal challenges that impact your well-being or ability to thrive at Binghamton University.
Accessing them, especially early on, as symptoms develop, can help support your academic success as a University student.

In the event I think you could benefit from such support, I will express my concerns (and the reasons for them) to you and remind you of our resources.
While I do not need to know the details of what is going on for you, your ability to share some of your situations with me will help me connect you with the appropriate support.

\section{Class Attendance Requirement}

Attendance is required and attendance will be checked regularly.
If you are not present when attendance is checked, it will be counted as missing the class.
Showing up late is considered missing the class.

\section{Communication}

Students must use their Binghamton email address for all course communication.
Emails from non-binghamton.edu domains will not receive a response.

All emails must include ``[CS415]'' or ``[CS515]'' in the subject line for proper identification.
Emails without this subject line identifier may be ignored.

I will make every effort to respond to student emails within two business days.
Please plan accordingly—emails sent shortly before deadlines may not receive timely responses.


\end{document} %}}}
%}}}